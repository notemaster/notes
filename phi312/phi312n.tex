% Created 2019-09-27 Fri 19:18
% Intended LaTeX compiler: pdflatex
\documentclass[11pt]{article}
\usepackage[utf8]{inputenc}
\usepackage[T1]{fontenc}
\usepackage{graphicx}
\usepackage{grffile}
\usepackage{longtable}
\usepackage{wrapfig}
\usepackage{rotating}
\usepackage[normalem]{ulem}
\usepackage{amsmath}
\usepackage{textcomp}
\usepackage{amssymb}
\usepackage{capt-of}
\usepackage{hyperref}
\usepackage{amssymb}
\author{Isaac Velasquez}
\date{\today}
\title{Intermediate Logic Notes}
\hypersetup{
 pdfauthor={Isaac Velasquez},
 pdftitle={Intermediate Logic Notes},
 pdfkeywords={},
 pdfsubject={},
 pdfcreator={Emacs 26.3 (Org mode 9.1.9)}, 
 pdflang={English}}
\begin{document}

\maketitle
\tableofcontents


\section{Introduction to Course 09/12/19}
\label{sec:orge593f7d}
\subsection{Administrative stuff}
\label{sec:orgb422adb}
Sign up for precept; Thursday times

\begin{enumerate}
\item Contemporary symbolic logic; Can start proving things; can write explicitly steps in math.
\item Ask questions about Logic can do; Metalogic/Metamathematics
\begin{enumerate}
\item Logic \(\sim\) Phi 201
\item Metalogic \(\sim\) Phi 312
\begin{enumerate}
\item Completeness and Soundness Proofs
\end{enumerate}
\end{enumerate}
\end{enumerate}
Emphasis is on \emph{perseverance}
Strongly encourage collaboration; Straight on definitions. Struggle
Math safe and not complete(Hilbert) or open and complete(Godel).


First couple of weeks is about propositional logic. 
theories in propositional logic \(\sim\) boolean algebra \(\sim\) Stone topological space

Predicate logic; Completeness, soundness
\subsection{Theories in the wild}
\label{sec:orgcde898d}

\begin{itemize}
\item Evolutionary Biology; Theories can be facts; Theory=System of beliefs to understand the world.
\item Relativity
\item Neuroscience
\item Folk Psycholgy; (think have understanding of thoughts and feelings)
\end{itemize}

\subsection{Regimented Theories (math)}
\label{sec:org0bacb72}

\begin{itemize}
\item Peano arithmetic
\item Group theory
\item Field Theory
\item Theory of Topological Spaces
\item Set Theory (Most general)
\item Elemental Thoery of the Category Sets
\end{itemize}

\subsection{How to decide which theory to believe}
\label{sec:org4076328}

Not a class on epistemology. Question before knowledge.
Suppose need two people, German and French. Both say Einstein's theories of relativity in their own
languages. This is a demonstration of equivalence.

\subsubsection{Metaconcept on how Theories Relate}
\label{sec:org12cc6f6}
Idea that some theories can be equivalent. 
e.g. Schrodinger(Wave mechanics) and Heisenberg(matrix mechanics). Bohr realizes that they are saying the 
same thing. Neumann proves that both theories are equivalent.

e.g. Folk psychology is \emph{reducible} to neuroscience.

Theory of heat; Thermodynamics is reducible to statistical mechanics.

\subsubsection{Explicating Terms}
\label{sec:org055f8be}
What we would like is precision in terms. Math likes to swallow things wwe already know
and be rigorous in definition. 

\begin{itemize}
\item \textbf{Continuous:}
\item \textbf{Equivilant}
\item \textbf{Reducible}
\item \textbf{Function}
\item \textbf{infinite}
\end{itemize}

Regimented Theory are theories in First Order Logic. Propositional Theories is subcollection.
Structure of Bolean Algebras \(\sim\) Stone Topological Space 
\section{Rigorous idea of the Theory 09/17/19}
\label{sec:org456e3ef}
\subsection{Pset 1 DUE DATE DEADLINE: \textit{<2019-09-23 Mon>}}
\label{sec:org613ee1a}
\subsection{Notion of the Theory; How to make it rigorous}
\label{sec:orgdff033c}
Define Theory in first-order logic. Subclass of Propositional Theories (Think of this as a network).
Need to understand the form, \emph{relations between theories.} Can study closeness of theories?

\textbf{Primary Focus:} Relation of theories, that we can schematize by \(\rightarrow\) with \(\rightarrow\) meaning \emph{translation}

\textbf{Equivalent:} Translation between theories

\textbf{Reduction:} Some kind of translation.

\begin{enumerate}
\item What is Propositional Logic
\item Define notion of translation between theories
\end{enumerate}
\subsection{Propositional Logic}
\label{sec:org79f0f84}
\begin{enumerate}
\item Grammar
\item Proof Theory
\item Semantics (Representation Theory)
\end{enumerate}

Hans is already using a Metatheory that is similar to Set Theory.
Logic has a grammar; A set of Symbols.

\(\Sigma\) = \{p\} where \(\Sigma\) is a non logical symbol

Logical Symbols don't have meaning, just help to make sentences;
\begin{enumerate}
\item \(\wedge\)
\item \(\vee\)
\item \(\rightarrow\)
\item etc.
\end{enumerate}

\textbf{Definition}: Let Str(\(\Sigma\)) be the set of finite strings
of Symbols: Logical + \(\Sigma\).

Literal Identity of Symbols. Sameness means same pattern. \(\sim\) Platonic

\(\phi\) is being used a metatheoretical tool.

\textbf{All the stuff below is Grammar}\\
\textbf{Definition:} The set Sent(\(\Sigma\)) is defined inductively by the following
\begin{enumerate}
\item \(\forall\) \(\phi\) \(\in\) \(\Sigma\), we have \(\phi\) \(\in\) Sent(\(\Sigma\))
\item \(\phi\) \(\in\) Sent(\(\Sigma\)) \(\rightarrow\) \textlnot{} \(\phi\) \(\in\) Sent(\(\Sigma\))
\item\relax [\(\phi\) \(\in\) Sent(\(\Sigma\)) \(\wedge\) \(\psi\) \(\in\) Sent(\(\Sigma\))] \implies \(\phi\) \(\rightarrow\) \(\psi\) \(\in\) Sent(\(\Sigma\)).
\item \(\wedge\)
\item \(\vee\)
\item No infinitely long sentences.
\end{enumerate}

There is at least one inductive set in the theory that we are looking at.

\subsection{Proof Theory}
\label{sec:org0ff94bc}

\(\Delta\) \vdash \(\phi\)

\(\Delta\), \(\Gamma\) = \(\Gamma\), \(\Delta\) where the comma acts as union

\begin{enumerate}
\item \(\phi\) \vdash \(\phi\) Rule of Assumptions.
\item \(\Delta\) \vdash \(\phi\) \(\wedge\) \(\Delta_0 \subseteq \Delta\) \(\rightarrow\) \(\Delta_0\) \vdash \(\phi\)
\end{enumerate}

\begin{center}
\begin{tabular}{ll}
\(\Gamma\) \vdash \(\phi\) \(\rightarrow\) \(\psi\) & \(\Delta\) \vdash \(\phi\)\\
\hline
\(\Gamma\) , \(\Delta\) \vdash \(\psi\) & \\
\end{tabular}
\end{center}


\begin{center}
\begin{tabular}{l}
\(\Gamma\) , \(\phi\) \vdash \(\psi\)\\
\hline
\(\Gamma\) \vdash \(\phi\) \(\rightarrow\) \(\psi\)\\
\end{tabular}
\end{center}


\begin{center}
\begin{tabular}{lll}
\(\Gamma\) \vdash \(\phi\) \(\vee\) \(\psi\) & \(\Delta\) , \(\phi\) \vdash \(\chi\) & \(\Theta\) , \(\psi\) \vdash \(\chi\)\\
\hline
 & \(\Gamma\) , \(\Delta\) , \(\Theta\) \vdash \(\chi\) & \\
\end{tabular}
\end{center}


\bot = contradiction

\begin{center}
\begin{tabular}{l}
\(\Gamma\) , \(\phi\) \vdash \bot\\
\hline
\(\Gamma\) \vdash \textlnot{} \(\phi\)\\
\end{tabular}
\end{center}
\subsection{Semantics}
\label{sec:org9009a22}

\textbf{Definition:} A \(\Sigma\) -valuation v is a function from \(\Sigma\) to \{0,1\}

\textbf{Fact} Each valuation v extends uniquely to a function \(\bar{v}\): Sent(\(\Sigma\)) \(\rightarrow\) \{0,1\}
s.t.  \(\bar{v}\) (\(\phi\)) = v(\(\phi\)), all \(\phi\) \(\in\) \(\Sigma\) \\
 \(\bar{v}\)  (\(\phi\) \(\wedge\) \(\psi\)) = min \{ \(\bar{v}(\phi)\) , \(\bar{v} (\psi)\) \}\\
 \(\bar{v}\) (\(\phi\) \(\vee\) \(\psi\)) = max\{ " "\} \\
 \(\bar{v}\) (\textlnot{} \(\phi\)) = 1 - \(\bar{v}(\phi)\) \\

\textbf{Definition}: \(\phi\) 
\begin{itemize}
\item contingent: One valuation  = 0 and one valuation = 1; At least
\item Tautologies: v(\(\phi\) ) = 1
\item Inconsistent: v(\(\phi\) ) = 0
\end{itemize}

Definition: \(\Delta\) \vDash \(\phi\) \iff \(\forall\) valuations v , v(\(\psi\) ) = 1 , all \(\psi\) \(\in\) \(\Delta\) , \(\rightarrow\) v(\(\phi\) ) = 1.

\(\vDash\) = Semantically implies

\textbf{Fact:} \(\phi\) \vDash \(\psi\) \(\wedge\) \(\psi\) \vDash \(\chi\) \(\rightarrow\) \(\psi\) \vDash \(\chi\).

\section{Theorems and Theory 09/19/19}
\label{sec:org728871a}
Sent(\(\Sigma\))  has a p or a q.\\
Sentences finitely long
Context: set of sentences\\
, = union\\
\(\phi\) .. \(\phi\) sentences individ.\\

\(\Delta\) \vdash \(\phi\) ; Proven inductively

\(\Delta\) \vDash \(\phi\) ; defined by valuation using 1,0 over universal; Not effectively decidable;
 n elementary sentences (p, q , \ldots{}) 2\(^{\text{n}}\) evaluations.

\textbf{Soundness}: If \(\Delta\) \vdash \(\phi\) then \(\Delta\) \vDash \(\phi\)

predicate built up inductively.

\textbf{Completeness}: If \(\Delta\) \vDash \(\phi\) the \(\Delta\) \vdash \(\phi\)

Second much more difficult than the first; Says that there is a proof.

Consistency: \(\exists\) v; v(\(\phi\))=1, \(\forall\) \(\phi\) \(\in\) \(\Delta\)

\textbf{Compactness:} 
\begin{enumerate}
\item If every finite subset \(\Delta_0\) of \(\Delta\) is consistent(There is a valuation that is true), Then \(\Delta\) is consistent.
\item \emph{Corollary}: If \(\Delta\) \vDash \(\phi\) then there is a finite subset \(\Delta_0 \subseteq \Delta s.t. \Delta_0 \vDash \phi.\)
\item Why? If \(\Delta\) \vDash \(\phi\) then \(\Delta\) , \textlnot{} \(\phi\) is inconsistent.
\item Hence there is a finite subset \(\Delta_0  \subseteq \Delta s.t. \Delta_0 , \not \phi\) is inconsistent \(\Delta_0 \vDash \phi\)

\item \(P1\)There is more that 1 number.
\item \(P2\)There is more than 2 numbers.
\end{enumerate}
\vdots

C There are infinitely many numbers

Set of axioms form a theory.

Propositional Theory requires a language(choose how many symbols/propositional constants; Need to choose a signature)

Empty Theory: 
\begin{center}
\begin{tabular}{lll}
\(\Sigma\) = \{p\} & T = 0 with no axioms & Will only get tautologies\\
\(\Sigma\) = \{p, q\} & T = 0 with no axioms\\
\(\Sigma\) = \{q\} & T= 0 with no axioms\\
\end{tabular}
\end{center}


Theory is not a set; contains a signature and a set of axioms. Choice of language and some axioms.

\(T \vdash \phi\) If you add the axioms of T then you can derive \(\phi\).

\textbf{Definition} v is a \emph{model} of T. if v(\(\phi\)) = 1, \(\forall\) \(\phi\) \(\in\) T. 
A valuation that makes all the sentences in T true.

\textbf{Definition} T is consistent if T has a model

\(\Sigma\) = \({ p_0,p_1,... }\)

T = \({ p_0\rightarrow p_1, p_0\rightarrow p_2, ... }\)

Is there a model for T? Any valuation where \(p_0\) is false there is infinite possibilities for a model.
Any valuation where \(p_0\) is true only has one possibility to make a model because everything must be true.

\textbf{Definition} A theory T is \textbf{complete} if for any \(\phi\) \(\in\) Sent(\(\Sigma\))
either T \vdash \(\phi\)  or \(T \vdash \not \phi\)

Empty theories are incomplete as it doesn't tell you anything about the signature;

E.g. \(\Sigma\) = \{p\} ; T = 0

To make it complete make T = \{p\}; There is just one valuation.

This new T has exactly one model. Show that T \vdash \(\phi\) and \(T \vdash \not \phi\)

new T \textasciitilde{} \(T_v = {\phi \in Sent(\Sigma)| v(\phi) = 1}\) 
\[T_v \vdash \phi\] or \[T_v \vdash \not \phi\]

complete and consistent has only one theory.

\textbf{Properties of Theories:}
\begin{itemize}
\item Complete
\item having finitely many axioms
\begin{itemize}
\item Isn't a big deal, infinitely increasing
\end{itemize}
\item \(\Sigma\) is finite
\item How do we decide what is interesting? To decide must be in relation to relations between theories.
\end{itemize}

\textbf{Relations between theories}

\textbf{Definition} A reconstrual of \(\Sigma\) in \(\Sigma_0\) is a function \(f: \Sigma\rightarrow Sent(\Sigma_0)\)

\textbf{Fact:} A reconstrual f extends uniquely to a function \(f: Sent(\Sigma) \rightarrow Sent(\Sigma_0) s.t. f(\not \phi) = \not f(\phi)\) 
and so on for other logical connectives

\textbf{Definition} f is a /translation from T to T\(\prime\). f: T\(\rightarrow\) T\(\prime\) iff for all \(\phi\) \(\in\) Sent(\(\Sigma\)) if T \vdash \(\phi\) then T\(\prime\) \vdash f(\(\phi\))

\(\Sigma\) = \({p_0,p_1,...}\) T = 0

T \vdash \(\phi\) \\
\vDash \(\phi\)

T\vDash f(\(\phi\)); Reconstruals always take 

\textbf{Definition} a T-atom; Atom relative to a Theory.

\begin{center}
\begin{tabular}{ll}
\top & \\
 & \(\psi\)\\
\(\phi\) & Nothing lower than it\\
\bot & \\
\end{tabular}
\end{center}

Atom for a theory is a sentence that is consistnent for a theory,
\section{Structure of Propositional Theories 09/24/19}
\label{sec:orge7900f3}
How to organize propositional Theories. Within Proposition Theories we can have 
T \(\rightarrow\) T'. (translation)
T''.

Another example is to replace propositional theories with sets. Applies to anything 
in pure mathematics such as groups, \ldots{} ,.etc . Let's continue with Sets though:

A\(\rightarrow\) B. (Function from A to B, is A x B)
C

An assignment of elements in B to elements of A.

\begin{enumerate}
\item A \(\rightarrow\) (f) B\(\rightarrow\) (g)C\(\rightarrow\) (h) D
\begin{enumerate}
\item \((g \circ f)(a) = g(f(a))\) where \(\circ\) is composition
\item \(h \circ(g\circ f) = (h\circ g)\circ f\)
\item \(\circ\) is associative
\end{enumerate}
\end{enumerate}

\(1_A(a)=a_1 \forall a \in A\) (identity)

C\(\rightarrow\) (g)A\(\rightarrow\) (f) B

\(\uparrow\)

\(1_A\)\\
\(f \circ 1_A = f\)\\
\(1_A \circ g = g\)\\

Everything you say is a category; Objects/things and arrows, composition, associative, and identity

Examples of categories
\begin{itemize}
\item Groups
\item Vectors
\item Topological spaces
\item Boolean Algebra
\item Rings
\item Sheathes on X
\item Categories; Must start on Categories of small categories;
\end{itemize}

Equivilance properties
\begin{itemize}
\item Symmetry
\item Reflexivity
\item Transitivity
\end{itemize}

\subsection{Category of theories}
\label{sec:org26d6833}
\begin{itemize}
\item Objects: Proof Theory; T | (\(\Sigma\), T) \implies T\vdash \(\phi\) and T'\vdash f(\(\phi\))
\item Arrows:
\begin{itemize}
\item reconstrual f; T \(\rightarrow\) T'
\item reconstrual g; T' \(\rightarrow\) T
\item T \(\in\) \(\Sigma\) ; T' \(\in\) \(\Sigma \prime\)
\item f \(\simeq\) g iff T' \vdash f(\(\phi\)) \iff g(\(\phi\))
\begin{itemize}
\item Reflexivity is obvious
\item Symmetry is true by nature of \iff
\end{itemize}
\end{itemize}
\item Define Composition in this category
\item \(T_0 \rightarrow (f)(f') T \rightarrow (g)(g') T'\)
\begin{itemize}
\item \([g]\circ [f] = [g \circ f]\)
\item Need to show that If f \(\simeq\) f' and g\(\simeq\) g' then \(g\circ f \simeq g'\circ f'\)
\item Let \(\phi\) \(\in\) \(Sent(\Sigma_0)\). Show T \vdash g(f(\(\phi\))) \iff g'(f'(\(\phi\)))
\item T \vdash f(\(\phi\)) \iff f'(\(\phi\))
\item Since g\(\simeq\) g', T' \vdash g(f'(\(\phi\)))\iff g'(f'(\(\phi\))) (1)
\item g(f(\(\phi\))\iff f'(\(\phi\))) = g(f(\(\phi\))) \iff g(f'(\(\phi\)))
\begin{itemize}
\item Literal equality of sentences; Definition of reconstrual
\end{itemize}
\item Since g is a translation, T' \vdash g(f(\(\phi\)))\iff g(f'(\(\phi\))) (2)
\item T' \vdash g(f(\(\phi\))) \iff g'(f'(\(\phi\)))
\end{itemize}
\item What is the Identity \(1_T\)?
\begin{itemize}
\item Identity is strict, only takes you to the same thing
\end{itemize}
\end{itemize}

How to decide definition of a category thoeries is c

\subsection{Special kinds of Arrows}
\label{sec:org860ab9f}

\begin{itemize}
\item Monomorphism
\begin{itemize}
\item \(f:A\rightarrow B\) if for any two arrows \(g: X \rightarrow A\) and \(h: X\rightarrow A\) if \(f\circ g= f \circ h\) then \(g =h\)
\item monic
\end{itemize}
\item Epimorphism
\begin{itemize}
\item f:A\(\rightarrow\) B is epi iff \(\forall\) g: B\(\rightarrow\) C and h: B\(\rightarrow\) C. If \(g\circ f = h\circ f\) then g= h
\item If f is epi then f is onto. F is injective i.e. doesnt cover whole range
\begin{itemize}
\item function \(g(y_0)=0\) and \(h(y_0)=1\)
\item \(g\circ f = h\circ f\)
\end{itemize}
\end{itemize}
\item Isomorphism
\begin{itemize}
\item Suppose we have two objects A and B; with f A\(\rightarrow\) B and g B\(\rightarrow\) A
\item \(gf=1_A\)
\item \(fg=1_B\)
\end{itemize}
\end{itemize}

These don't always correspond set theoretic notions. How do these relate?

\begin{center}
\begin{tabular}{lll}
Term & Definition & Example\\
\hline
monomorphism/monic & f:A\(\rightarrow\) B if for any two arrows g: X \(\rightarrow\) A and h: X\(\rightarrow\) A if f\^{} g= f \^{} h then g =h$\backslash$) & \\
\end{tabular}
\end{center}

\begin{itemize}
\item Every isomorphism is a monomorphism and every iso is an epi
\item Try to do inverse; should fail.
\end{itemize}
\end{document}