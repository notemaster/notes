% Created 2019-09-28 Sat 21:08
% Intended LaTeX compiler: pdflatex
\documentclass[11pt]{article}
\usepackage[utf8]{inputenc}
\usepackage[T1]{fontenc}
\usepackage{graphicx}
\usepackage{grffile}
\usepackage{longtable}
\usepackage{wrapfig}
\usepackage{rotating}
\usepackage[normalem]{ulem}
\usepackage{amsmath}
\usepackage{textcomp}
\usepackage{amssymb}
\usepackage{capt-of}
\usepackage{hyperref}
\author{Isaac Velasquez}
\date{\today}
\title{}
\hypersetup{
 pdfauthor={Isaac Velasquez},
 pdftitle={},
 pdfkeywords={},
 pdfsubject={},
 pdfcreator={Emacs 26.3 (Org mode 9.1.9)}, 
 pdflang={English}}
\begin{document}

\tableofcontents

\section{Chapter 1}
\label{sec:org73310d3}
Trying to do a brief overview of some of the terms and important concepts
in the \emph{"Invitation to Metatheory"}

A reminder: Doing metatheory is about \emph{relations}. We are not studying Metatheory but using it
to talk about theories.

\begin{center}
\begin{tabular}{lll}
Vocabulary & Symbol & Definition\\
\hline
\textbf{Propositional Signature} & \(\Sigma\) & Collection of Elementary Sentences\\
\textbf{Context} & \(\phi_1\dots\phi_n\) OR \(\Delta\), \(\Gamma\) & Finite collection of Sentences/Sequences\\
\textbf{Derivability} & \vdash & Smallest relation between sets of Sent\\
 &  & Syntatically entails\\
\textbf{Proof} & \(\Delta\) \vdash \(\phi\) & \\
\textbf{Provable} & \vdash \(\phi\) OR \top & \\
\textbf{Not Provable} & \bot OR \(\not \top\) & \\
\textbf{Interpretation/Valuation} & f: \(\Sigma\) \(\rightarrow\) \{true, false\} & \\
\textbf{Propositional Theory} & T ; \(\Sigma\) \(\wedge\) \(\Delta\) \(\in\) \(\Sigma\) & \\
\textbf{Tarski Truth} & v(\(\phi\)) = 1 \(\rightarrow\) \(\phi\) is true in v & v = interp. of \(\Sigma\) and \(\phi\) = sent. \(\in\) \(\Sigma\)\\
\textbf{Model} & v(\(\phi\)) = 1 \(\forall\) \(\phi\) \(\in\) \(\Delta\) & v is a model of \(\Delta\) which is \(\in\) \(\Sigma\) sent.\\
\textbf{Consistent} & \(\Delta\) has at least one model v & \\
\textbf{Inconsistent} & \(\Delta\) has no models v & \\
\textbf{Proof of Relative Cons.} &  & X theory \(\rightarrow\) models of accepted Y\\
 &  & Y = Boolean Algebra / Set Theory\\
\textbf{Semantically entails} & v(\(\phi\)) = 1 \(\forall\) \(\phi\) \(\in\) \(\Delta\) \(\rightarrow\) \(\Delta\) \vDash \(\phi\) & \(\Delta\) = set of \(\Sigma\) sent.; \(\phi\) = \(\Sigma\) sent.\\
 & \(\Delta\) semantically entails \(\phi\) & \\
\textbf{Soundness} & \(\Delta\) \vdash \(\phi\) \(\rightarrow\) \(\Delta\) \vDash \(\phi\) & If proof then truth table to match\\
\textbf{Completeness} & \(\Delta\) \vDash \(\phi\) \(\rightarrow\) \(\Delta\) \vdash \(\phi\) & If truth table then must be proof\\
\textbf{Compactness} & \(\forall\) \(\Delta_F \subseteq \Delta\) cons.\(\rightarrow\) \(\Delta\) cons. & \(\Delta\) set of sentences. \(\Delta_F\) finite subset.\\
\textbf{Complete Theory} & \(\Delta cons. \wedge \forall \phi \in \Sigma , \Delta \vDash \phi \vee \Delta \vDash \not \phi\) & T consists of axioms \(\Delta\) \(\in\) \(\Sigma\)\\
\textbf{Deductive Closure} & Cn(T)=T\implies \{Sent(\(\Sigma\))\} & T theory in \(\Sigma\)\\
\textbf{Reconstrual} & f: \(\Sigma\) \(\rightarrow\) Sent(\(\Sigma \prime\)) & \\
\textbf{Substitution} & \(\forall f:\Sigma\rightarrow \Sigma\prime , \phi \vdash \psi \rightarrow f(\phi) \vdash f(\psi)\) & Where f is a reconstrual\\
\textbf{Translation} & T \vdash \(\phi\) \(\rightarrow\) T\(\prime\) \vdash f(\(\phi\)) & f is a special reconstrual\\
\textbf{Equality of Trans.} & \(T\prime \vdash f(p) \iff g(p) \forall p \in \Sigma\) & f \(\simeq\) g; f, g trans. from T to T\(\prime\)\\
\textbf{Homotopy Equiv.} & \(\exists\) f: T\(\rightarrow\) T\(\prime\) \(\wedge\) g: T\(\prime \rightarrow\) T s.t. gf \(\simeq\) 1\(_{\text{T}}\) \(\wedge\) fg \(\simeq\) 1\(_{\text{T}\prime}\) & \\
\end{tabular}
\end{center}



Two Propositional Constants are defined by identity and are dependent
on the signature they are a part of.

\textbf{Bare Set} is related to the Leibniz's identity of indiscernibles which states that
there cannot be separate objects or enitites that have all their properites in common.
What this means, is essentially that a Bare Set for a Propositional Signature just means
that there can be separate objects with all the same common properties. Nothing stands closer
together in relation

The Possible set of Sentences that can be formed from \(\Sigma\) is called
the set Sent(\(\Sigma\)). These rules are the following:

\begin{enumerate}
\item \(\phi\) \(\in\) \(\Sigma\) \(\rightarrow\) \(\phi\) \(\in\) Sent(\(\Sigma\))
\item \(\phi\) \(\in\) Sent(\(\Sigma\)) \(\rightarrow\) (\textlnot{} \(\phi\)) \(\in\) Sent(\(\Sigma\))
\item \(\phi\) \(\in\) Sent(\(\Sigma\)) \(\wedge\) \(\psi\) \(\in\) Sent(\(\Sigma\)) \(\rightarrow\) (\(\phi\) \(\wedge\) \(\psi\)) \(\in\) Sent(\(\Sigma\)), (\(\phi\) \(\vee\) \(\psi\)) \(\in\) Sent(\(\Sigma\)), \(\wedge\)  (\(\phi\) \(\rightarrow\) \(\psi\)) \(\in\) Sent(\(\Sigma\))
\item Nothing is in Sent(\(\Sigma\)) unless it enters one of the previous clauses
\end{enumerate}

\textbf{Sentence:} Finite string of symbols and contains finitely many propositional constants.

Unions are commutative in Contexts.
\subsection{Proof by Induction}
\label{sec:orga34768e}
\begin{enumerate}
\item Show that some property of interest P, holds of the elements of \(\Sigma\)
\item Show that [P holds of \(\phi\) \(\rightarrow\) P holds of \(\psi\)]
\item Show if P holds of \(\phi\) and \(\psi\) \(\rightarrow\) P holds of \(\phi\) \(\wedge\) \(\psi\), \(\phi\) \(\vee\) \(\psi\), and \(\phi\) \(\rightarrow\) \(\psi\)
\end{enumerate}

We allow for empty contexts as well

\subsection{Rules for Turnstile}
\label{sec:orgb115617}
\begin{enumerate}
\item \(\vdash\) is closed under the previously given logical clauses
\item \(\Delta\) \vdash \(\phi\) and \(\Delta\) \subseteq \(\Delta_0\) , then \(\Delta_0\) \vdash \(\phi\) . (\textbf{Monotonicity})
\end{enumerate}

\(\Delta\) , \(\phi\) \vdash \bot \(\rightarrow\) \(\Delta\) \vdash \(\not \phi\)

Semantics fundamentally differs from Syntax by introducing the concept of Truth 
and false; Essentially, introducing a world

Interpretation v of \(\Sigma\) extends to a function v: Sent(\(\Sigma\)) \(\rightarrow\) \{0,1\}
by the following clauses:
\begin{enumerate}
\item v(\textlnot{} \(\phi\)) = \iff v(\(\phi\)) = 0
\item v(\(\phi\) \(\wedge\) \(\psi\)) = 1 \iff v(\(\phi\)) = 1 \(\wedge\) v(\(\psi\)) = 1.
\item v(\(\phi\) \(\vee\) \(\psi\)) = 1 \iff v(\(\phi\)) = 1 \(\vee\) v(\(\psi\)) = 1
\item v(\(\phi\) \(\rightarrow\) \(\psi\)) = v(\textlnot{} \(\phi\) \(\vee\) \(\psi\)).
\end{enumerate}

Sent(\(\Sigma\)) could be thought of as simply sentences using the vocabulary of the \(\Sigma\) to create sentences without
being interpreted. When we make it into a function/interpretation v where the set of sentences is mapped
to a set \{1,0\}, we could say that it then endows the senteces/symbols with meaning due to our judgement on these
sentences. However, it is important to note that the domain of predicate logic interpretation must be a \textbf{set}. End stop. This
is something that can be demonstrated using set theory. The World is not a consequence of set theory \(\therefore\) the world is not a set.
They are of two different qualities.

\begin{center}
\begin{tabular}{ll}
Abstract & Concrete\\
\hline
Set & World\\
\end{tabular}
\end{center}

Can write T instead of \(\Delta\), but the formulation of \(\Sigma\) affects T.\\
e.g. p would be different in \(\Sigma\) = \{p\} than in \(\Sigma\) = \{p,q\}

A Concept defined for sets of sentences such as consistency can also apply
to Theories but, \emph{again}, it must be emphasized that a theory is always the same:
\[T = {\Sigma , \Delta \in \Sigma}\]

Where \(\Sigma\) is a Propositional Sig. and \(\Delta\) are a set of sentences in that signature.

\subsection{Exercise 1.3.8}
\label{sec:org23e4369}


\textbf{Question} For \(\Delta\) a set of \(\Sigma\) sentences and for \(\Sigma\) sentences \(\phi\) and \(\psi\), if \(\Delta\), \(\phi\) \vDash \(\psi\)
then \(\Delta\) \vDash \(\phi\) \(\rightarrow\) \(\psi\)

Assume \(\Delta\) , \(\phi\) \vDash \(\psi\) and that \(\exists\) v \(\in\) \(\Sigma\) s.t. \(\forall\) \(\chi\) \(\in\) \(\Delta\) : 
v(\(\chi\)) = 1 and v(\(\phi\) \(\rightarrow\) \(\psi\)) = 0

By the material conditional we can do the following move

This thereby leaves us with the following result:

v(\(\phi\)) = 1 and v(\(\psi\)) = 0

However, we assumed \(\Delta\), \(\phi\) \vDash \(\psi\) which is impossible if
\(\psi\) = 0. Therefore, there is a contradiction. Q.E.D \square

\textbf{Another Solution:}
Suppose that \(\Delta\), \(\phi\) \vDash \(\psi\). We want to show that any model v of \(\Delta\) is such that
v(\(\phi\) \(\rightarrow\) \(\psi\)) = 1, i.e.,
\[v(\not \phi) = 1 or v(\psi) = 1\]

So, given any such interpretation v. Either (i) v(\(\phi\)) = 1 or (ii) v(\textlnot{} \(\phi\)) = 1.
If (i), then v is a model of \(\Delta\), \(\phi\), since it is a model of \(\Delta\). Hence by our original assumption and
the definition of semantic entailment v(\(\psi\)) = 1 and we are done, since (\textbf{) follows.
If (ii), then we are also done, since (}) follows. \square

\subsection{Exercise 1.3.9 **}
\label{sec:org918cb56}
Show that \(\Delta\) \vDash \(\phi\) \iff \(\Delta\) \(\cup\) \{\textlnot{} \(\phi\) \} is inconsistent. Here
\(\Delta\) \(\cup\) \{\textlnot{} \(\phi\)\} is the theory consisting of \textlnot{} \(\phi\) and all sentences in \(\Delta\).

Suppose that \(\Delta\) \vDash \(\phi\) and \(\Delta\) \(\cup\) \{\textlnot{} \(\phi\)\} is consistent. This means
that we can find a model v for \(\Delta\) s.t. v(\(\chi\))=1 \(\forall\) \(\chi\) \(\in\) \(\Delta\). This would
mean that both v(\(\phi\)) and v(\textlnot{} \(\phi\)) = 1. However, this isn't possible given the fact
that one of the clauses of an interpretation is that v(\textlnot{} \(\phi\))=1 \iff v(\(\phi\)) = 1 contradicting
our initial assumption.\square
\subsection{Parameters for Reconstruals}
\label{sec:orgb9810b8}

\begin{enumerate}
\item for p in \(\Sigma\), \(\bar{f} (p) = f(p)\)
\item \(\forall\) \(\phi\), \(\bar{f}(\not \phi) = \not \bar{f}(\phi)\)
\item binary circles can be taken out to function f
\end{enumerate}

\subsection{Exercise 1.3.16 **}
\label{sec:org0af5b97}
Let T\(\prime\) = \{p\} \(\in\) \(\Sigma \prime\) and \(\Sigma \prime\) = \{p,q\}. Show that T\(\prime\) is not complete.

Proof by Contradiction:
Assume that T\(\prime\) is complete, this means that T\(\prime\) is consistent(\(\Delta\) has at least one model) and that
\(\forall \phi \in \Sigma\prime;\Delta \vDash \phi \vee \Delta \vDash \not \phi\).
However, T\(\prime\) only describes p,  so although we can say \(\Delta \vDash p\), 
we cannot say anything about q in relation to \(\Delta\) which means we can't say that \(\Delta \vDash q\) or \(\Delta\)\vDash \textlnot{} q 

\subsection{Exercise 1.3.17}
\label{sec:org711e317}
Show that Cn(Cn(T)) = Cn(T).

Cn(T) is a deductive closure which is the set of \(\Sigma\) sentences that is implied by T.

\subsection{Exercise 1.4.7}
\label{sec:org6e72c7d}
Prove that if v is a model of T\(\prime\) , and f:T\(\rightarrow\) T\(\prime\) is a translation, then
\(v \circ f\) is a model of T. Here \(v \circ f\) is the interpretation of \(\Sigma\) optained by applying
f first, and then applying v.


\subsection{Lingering Questions:}
\label{sec:org3cc7420}
What does it mean for \(\Delta\) to have more than one model?

\section{Week 2}
\label{sec:orgf6ceb2b}
\subsection{Sets is a Category (29)}
\label{sec:org5ebc5ac}
\begin{center}
\begin{tabular}{lll}
Terms & Symbology & Definition\\
Category &  & Objects and arrows\\
Monomorphism & g: Z\(\rightarrow\) X & fg=fh\(\rightarrow\) g=h\\
 & h: Z\(\rightarrow\) X & where f: X\(\rightarrow\) Y\\
Epimorphism & g: Y\(\rightarrow\) Z & gf=hf, then g=h\\
 & h: Y\(\rightarrow\) Z & \\
Isomorphism & \(\exists\) g:Y\(\rightarrow\) X & gf=\(1_X\) and fg=\(1_Y\)\\
 &  & X\(\simeq\) Y\\
\textbf{THEORIES} &  & \\
Conservative & T'\vdash f(\(\phi\))\(\rightarrow\) T\vdash\(\phi\) & \(\forall\) \(\phi\) \(\in\) Sent(\(\Sigma\))\\
 &  & \\
Ess. surj. & T'\vdash \(\phi\) \iff f(\(\psi\)) & \(\forall \phi \in \Sigma^\prime; \exists \psi \in \Sigma\)\\
 &  & \\
\end{tabular}
\end{center}

We are going to be looking at the general theory of categories. 
To do this we are going to look at Set Theory as an example. In this case,
the theory of sets has objects called sets and relations called functions.

What does it mean to say a set is category? It means the following

\begin{enumerate}
\item Every function f has a domain set \(d_0f\) and a codomain set \(d_1f\).
\begin{enumerate}
\item \(f:X\rightarrow Y\)indicates that \(d_0f=X\) \(d_1f=Y\)
\end{enumerate}
\item Compatible functions can be composed.
\begin{enumerate}
\item e.g. can have transitivity f: x\(\rightarrow\) y g: y\(\rightarrow\) z \(f \circ g: x\rightarrow z\)
\item These compositions can also be associative
\item \(h \circ (g \circ f) = (h \circ g) \circ f)\)
\end{enumerate}
\item \(\forall\) X | X is a set, there is a function \(1_X:X\rightarrow X\) that acts as a left and right identity relative to composition.
\end{enumerate}

In short, there are objects, arrows, composition with transitivity and associativity, and identity which should 
conform to equation.

\textbf{Containment or \(\in\) is \emph{not} a primitive notion of the Elemental Theory of the Category of Sets}

\subsection{Chapter 3.1 On the Category of Propositional Theories}
\label{sec:orge0d4fd9}
\textbf{Th} is the category of propositional Theories

f:T\(\rightarrow\) T'\\
g:T\(\rightarrow\) T'\\
f and g are equal, f\(\simeq\) g, just in case T' \vdash f(\(\phi\)) \iff g(\(\phi\)) \(\forall\) \(\phi\) \(\in\) 
Sent(\(\Sigma\))

\subsection{Proving that conservative translations are monomorphic}
\label{sec:orgb268540}
Suppose first that f is conservative with two granslations g, h from T'' \rightarro T s.t.
\(f\circ g = f\circ h\).
\end{document}